% Options for packages loaded elsewhere
\PassOptionsToPackage{unicode}{hyperref}
\PassOptionsToPackage{hyphens}{url}
%
\documentclass[
  english,
  man, donotrepeattitle,floatsintext]{apa6}
\title{ManyBabies2 Supplemental Material}
\author{\phantom{0}}
\date{}

\usepackage{amsmath,amssymb}
\usepackage{lmodern}
\usepackage{iftex}
\ifPDFTeX
  \usepackage[T1]{fontenc}
  \usepackage[utf8]{inputenc}
  \usepackage{textcomp} % provide euro and other symbols
\else % if luatex or xetex
  \usepackage{unicode-math}
  \defaultfontfeatures{Scale=MatchLowercase}
  \defaultfontfeatures[\rmfamily]{Ligatures=TeX,Scale=1}
\fi
% Use upquote if available, for straight quotes in verbatim environments
\IfFileExists{upquote.sty}{\usepackage{upquote}}{}
\IfFileExists{microtype.sty}{% use microtype if available
  \usepackage[]{microtype}
  \UseMicrotypeSet[protrusion]{basicmath} % disable protrusion for tt fonts
}{}
\makeatletter
\@ifundefined{KOMAClassName}{% if non-KOMA class
  \IfFileExists{parskip.sty}{%
    \usepackage{parskip}
  }{% else
    \setlength{\parindent}{0pt}
    \setlength{\parskip}{6pt plus 2pt minus 1pt}}
}{% if KOMA class
  \KOMAoptions{parskip=half}}
\makeatother
\usepackage{xcolor}
\IfFileExists{xurl.sty}{\usepackage{xurl}}{} % add URL line breaks if available
\IfFileExists{bookmark.sty}{\usepackage{bookmark}}{\usepackage{hyperref}}
\hypersetup{
  pdftitle={ManyBabies2 Supplemental Material},
  pdflang={en-EN},
  hidelinks,
  pdfcreator={LaTeX via pandoc}}
\urlstyle{same} % disable monospaced font for URLs
\usepackage{graphicx}
\makeatletter
\def\maxwidth{\ifdim\Gin@nat@width>\linewidth\linewidth\else\Gin@nat@width\fi}
\def\maxheight{\ifdim\Gin@nat@height>\textheight\textheight\else\Gin@nat@height\fi}
\makeatother
% Scale images if necessary, so that they will not overflow the page
% margins by default, and it is still possible to overwrite the defaults
% using explicit options in \includegraphics[width, height, ...]{}
\setkeys{Gin}{width=\maxwidth,height=\maxheight,keepaspectratio}
% Set default figure placement to htbp
\makeatletter
\def\fps@figure{htbp}
\makeatother
\setlength{\emergencystretch}{3em} % prevent overfull lines
\providecommand{\tightlist}{%
  \setlength{\itemsep}{0pt}\setlength{\parskip}{0pt}}
\setcounter{secnumdepth}{-\maxdimen} % remove section numbering
% Make \paragraph and \subparagraph free-standing
\ifx\paragraph\undefined\else
  \let\oldparagraph\paragraph
  \renewcommand{\paragraph}[1]{\oldparagraph{#1}\mbox{}}
\fi
\ifx\subparagraph\undefined\else
  \let\oldsubparagraph\subparagraph
  \renewcommand{\subparagraph}[1]{\oldsubparagraph{#1}\mbox{}}
\fi
\newlength{\cslhangindent}
\setlength{\cslhangindent}{1.5em}
\newlength{\csllabelwidth}
\setlength{\csllabelwidth}{3em}
\newlength{\cslentryspacingunit} % times entry-spacing
\setlength{\cslentryspacingunit}{\parskip}
\newenvironment{CSLReferences}[2] % #1 hanging-ident, #2 entry spacing
 {% don't indent paragraphs
  \setlength{\parindent}{0pt}
  % turn on hanging indent if param 1 is 1
  \ifodd #1
  \let\oldpar\par
  \def\par{\hangindent=\cslhangindent\oldpar}
  \fi
  % set entry spacing
  \setlength{\parskip}{#2\cslentryspacingunit}
 }%
 {}
\usepackage{calc}
\newcommand{\CSLBlock}[1]{#1\hfill\break}
\newcommand{\CSLLeftMargin}[1]{\parbox[t]{\csllabelwidth}{#1}}
\newcommand{\CSLRightInline}[1]{\parbox[t]{\linewidth - \csllabelwidth}{#1}\break}
\newcommand{\CSLIndent}[1]{\hspace{\cslhangindent}#1}
% Manuscript styling
\usepackage{upgreek}
\captionsetup{font=singlespacing,justification=justified}

% Table formatting
\usepackage{longtable}
\usepackage{lscape}
% \usepackage[counterclockwise]{rotating}   % Landscape page setup for large tables
\usepackage{multirow}		% Table styling
\usepackage{tabularx}		% Control Column width
\usepackage[flushleft]{threeparttable}	% Allows for three part tables with a specified notes section
\usepackage{threeparttablex}            % Lets threeparttable work with longtable

% Create new environments so endfloat can handle them
% \newenvironment{ltable}
%   {\begin{landscape}\centering\begin{threeparttable}}
%   {\end{threeparttable}\end{landscape}}
\newenvironment{lltable}{\begin{landscape}\centering\begin{ThreePartTable}}{\end{ThreePartTable}\end{landscape}}

% Enables adjusting longtable caption width to table width
% Solution found at http://golatex.de/longtable-mit-caption-so-breit-wie-die-tabelle-t15767.html
\makeatletter
\newcommand\LastLTentrywidth{1em}
\newlength\longtablewidth
\setlength{\longtablewidth}{1in}
\newcommand{\getlongtablewidth}{\begingroup \ifcsname LT@\roman{LT@tables}\endcsname \global\longtablewidth=0pt \renewcommand{\LT@entry}[2]{\global\advance\longtablewidth by ##2\relax\gdef\LastLTentrywidth{##2}}\@nameuse{LT@\roman{LT@tables}} \fi \endgroup}

% \setlength{\parindent}{0.5in}
% \setlength{\parskip}{0pt plus 0pt minus 0pt}

% Overwrite redefinition of paragraph and subparagraph by the default LaTeX template
% See https://github.com/crsh/papaja/issues/292
\makeatletter
\renewcommand{\paragraph}{\@startsection{paragraph}{4}{\parindent}%
  {0\baselineskip \@plus 0.2ex \@minus 0.2ex}%
  {-1em}%
  {\normalfont\normalsize\bfseries\itshape\typesectitle}}

\renewcommand{\subparagraph}[1]{\@startsection{subparagraph}{5}{1em}%
  {0\baselineskip \@plus 0.2ex \@minus 0.2ex}%
  {-\z@\relax}%
  {\normalfont\normalsize\itshape\hspace{\parindent}{#1}\textit{\addperi}}{\relax}}
\makeatother

\makeatletter
\usepackage{etoolbox}
\patchcmd{\maketitle}
  {\section{\normalfont\normalsize\abstractname}}
  {\section*{\normalfont\normalsize\abstractname}}
  {}{\typeout{Failed to patch abstract.}}
\patchcmd{\maketitle}
  {\section{\protect\normalfont{\@title}}}
  {\section*{\protect\normalfont{\@title}}}
  {}{\typeout{Failed to patch title.}}
\makeatother

\usepackage{xpatch}
\makeatletter
\xapptocmd\appendix
  {\xapptocmd\section
    {\addcontentsline{toc}{section}{\appendixname\ifoneappendix\else~\theappendix\fi\\: #1}}
    {}{\InnerPatchFailed}%
  }
{}{\PatchFailed}
\DeclareDelayedFloatFlavor{ThreePartTable}{table}
\DeclareDelayedFloatFlavor{lltable}{table}
\DeclareDelayedFloatFlavor*{longtable}{table}
\makeatletter
\renewcommand{\efloat@iwrite}[1]{\immediate\expandafter\protected@write\csname efloat@post#1\endcsname{}}
\makeatother
\usepackage{lineno}

\linenumbers
\usepackage{csquotes}
\ifXeTeX
  % Load polyglossia as late as possible: uses bidi with RTL langages (e.g. Hebrew, Arabic)
  \usepackage{polyglossia}
  \setmainlanguage[]{english}
\else
  \usepackage[main=english]{babel}
% get rid of language-specific shorthands (see #6817):
\let\LanguageShortHands\languageshorthands
\def\languageshorthands#1{}
\fi
\ifLuaTeX
  \usepackage{selnolig}  % disable illegal ligatures
\fi


\shorttitle{Supplemental Material}

\affiliation{\phantom{0}}

\begin{document}
\maketitle

This document contains supplemental material of the manuscript:

Schuwerk, T.*, Kampis, D.*, Baillargeon, R., Biro, S., Bohn, M., Byers-Heinlein, K., Dörrenberg, S., Fisher, C., Franchin, L., Fulcher, T., Garbisch, I., Geraci, A., Grosse Wiesmann, C., Hamlin, J. K., Hepach, R., Hunnius, S., Hyde, D. C., Kármán, P., Kosakowski, H. L., Kovács, Á. M., Krämer, A., Kulke, L., Lee, C., Lew-Williams, C., Liszkowski, U., Mahowald, K., Mascaro, O., Meyer, M., Moreau, D., Perner, J., Poulin-Dubois, D., Powell, L. J., Prein, J., Priewasser, B., Proft, M., Raz, G., Reschke, P., Ross, J., Rothmaler, K., Saxe, R., Schneider, D., Southgate, V., Surian, L., Tebbe, A.-L., Träuble, B., Tsui, A. S. M., Wertz, A. E., Woodward, A., Yuen, F., Yuile, A. R., Zellner, L., Frank, M.C., \& Rakoczy, H. (2021, February 14). Action anticipation based on an agent's epistemic state in toddlers and adults. {[}Manuscript submitted for publication{]} (*shared co-first authorship).

\hypertarget{s1.-pilot-studies}{%
\section{S1. Pilot Studies}\label{s1.-pilot-studies}}

The familiarization trials were developed to convey information that is necessary for correct action predictions in this paradigm. First, the agent's goal is introduced, i.e.~the chaser wants to catch their partner (the chasee). Second, the situational constraints of the scene are shown. A barrier (fence) divides the scene so that the other side can only be reached by going through a y-shaped tunnel. Yet, it had to be clear that the fence is not a visual barrier, meaning that the chaser can see everything that takes place on the other side. Third, the familiarization trials should teach the timing of events, particularly, how much time the chaser spends in the tunnel and when their reappearance is to be expected. We piloted the stimuli with adults and toddlers between 18 and 27 months of age, the core age range of our main study. All analysis scripts can be found on GitHub (\url{https://github.com/manybabies/mb2-analysis}).

\hypertarget{pilot-1}{%
\subsection{Pilot 1}\label{pilot-1}}

In the first pilot study, we wanted to get an estimate of the level of correct goal-based action predictions with these novel stimuli. We presented a total of eight familiarization trials. An observation of changes in the anticipation rate over trials would help us to determine the optimal number of familiarization trials. Further, we used this pilot to test the general procedure (i.e., data collection in different labs, preprocessing and analysis of raw gaze data from different eye-trackers). We also checked whether gaze patterns indicated any issues with perceptual properties of stimuli, such as distracting visual saliencies. Data for this pilot study was collected between February and July 2019.

\hypertarget{methods}{%
\subsubsection{Methods}\label{methods}}

\hypertarget{participants.}{%
\paragraph{Participants.}\label{participants.}}

Seven labs\footnote{The contributing labs were: CEU Cog Dev Center, Central European University, Budapest; Babylab Copenhagen, University of Copenhagen, Denmark; Göttinger Kindsköpfe, Georg-August-Universität Göttingen, Germany; LMU Babylab, Ludwig-Maximilians-Universität München, Germany; Babylab Uni Trento, University of Trento, Italy; Center for Infant Cognition, University of British Columbia, Canada; Infant Learning and Development Lab, University of Chicago, USA} tested a total of 65 healthy full-term toddlers (28 males; Mean age = 23.14 months; range: 18.25 to 26.84 months). Data from eight additional toddlers were excluded from the analyses. Three did not complete the full experiment, another three did not complete at least six trials. Two toddlers had to be excluded due to technical problems with data collection (e.g., calibration of eye-tracker). At the trial level, four additional trials were excluded because the trial data was incomplete (as determined by not having at least 32 s of eye-tracking data for that trial, from the beginning to the end of the trial). A total of 42 adults were tested in three labs {[}5 males, 1 male/other, 1 N/C (not collected); Mean age = 24.10 years; range: 19 to 53 years{]}. One adult was excluded because this participant did not complete at least six trials. We asked contributing labs for a minimum sample size of 3-5 participants per age group. We reasoned that the resulting minimum total sample of 27-45 participants per age group would be large enough for an initial estimate of anticipatory looking (AL) behavior. The contributing labs were independently responsible for obtaining informed written consent and reimbursing participants. Each lab acquired ethics approval. Central data analyses only used de-identified data. Video recordings of participants were archived locally at each lab following the local data protection regulations.

\hypertarget{task-and-procedure.}{%
\paragraph{Task and Procedure.}\label{task-and-procedure.}}

Toddlers were tested in a quiet room of nurseries or laboratories, after their caregivers read and signed the informed consent form. They sat on an educator/caregiver's lap or on a car seat, centered in front of the monitor used to display the stimuli at a distance of about 60-80 cm. Educators or caregivers were instructed to remain silent and to wear black glasses or close their eyes to avoid erroneous tracking of their eyes. The experimenter was behind a curtain/room divider and controlled stimulus presentation. Depending on the lab setup, the following eye-tracking systems were used: Tobii T60 (two labs), Tobii T120 (two labs), EyeLink 1000 Plus (two labs), SMI250Redmobile (two labs), SMI iView X Hi-Speed 1250 (one lab). For each lab the following information was collected: type of eye-tracker apparatus, trial order condition (A or B), any procedural or technical error that occurred during the experimental session, location of the lab they were tested in (laboratory or nursery).
The task consisted of a calibration check, eight familiarization trials and another final calibration check. After an initial attention getter, participants were presented with the calibration check that consisted of an animated star with sound, moving and stopping at four locations. The familiarization trials were as described in the Methods section of the main study, with the following deviations: In the upper part of the tunnel there was a small window that allowed participants to watch the agents moving inside the upper part of the tunnel before it forked. Further, unlike in the final familiarization trial version, a chime sounded at the moment the chaser disappeared from the tunnel window, indicating the start of the anticipatory period. The starting location of the chasee (left or right half of the upper part of the scene) and the box the chasee ended up (left or right box) were counterbalanced, resulting in a total of four familiarization trial versions {[}started from the right and ended up in right box (RR); started from the right and ended up in left box (RL); started from the left and ended up in right box (LR); started from the left and ended up in left box (LL){]}. Each of these versions was presented twice in two pseudo-randomized orders (Order A: LL1, LR2, RR2, RR1, LL2, RL2, LR1, RL1; Order B: RL1 LR1, RL2, LL2, RR1, RR2, LR2, LL1). Half of the participants in each lab group were randomly assigned to one of the two orders.

\hypertarget{data-analysis.}{%
\paragraph{Data Analysis.}\label{data-analysis.}}

The labs exported the raw gaze data in the format the respective eye-tracking software allowed. The participants' demographic information and details about the test session were collected in standardized spreadsheets. Each lab provided the raw gaze data and de-identified demographic information with Google Drive. Data preprocessing was identical to the procedure of the current study. For details refer to the Methods section of the main manuscript.

\hypertarget{results}{%
\subsubsection{Results}\label{results}}

\hypertarget{descriptive-statistics.}{%
\paragraph{Descriptive Statistics.}\label{descriptive-statistics.}}

In Figure S1, we show the toddlers' proportion of first looks and the proportion looking at each of the critical AOIs (target, distractor, other) during the anticipatory period of each trial. Figure S2 (plots labeled pilot 1) shows the proportion of looking of toddlers and adults as a smooth curve, generated by binning the data and averaging the proportion looking at each time point across all participants. We saw robust evidence for looks to the target relative to the distractor during the anticipation period, as evidenced by the red lines being consistently higher than the blue lines. In Figure S2, we separated trials into two blocks (Trials 1-4 and Trials 5-8). For toddlers in Pilot 1, we see similar rates of anticipation for Trials 1-4, as in Trials 5-8. In fact, anticipation is slightly lower in Trials 5-8 than in Trials 1-4. For adults, we see an increase in the anticipation rate in Trials 5-8.
The heatmaps in Figure S3 illustrate the distribution of looks to scene locations during the anticipatory period. We found that a large proportion of anticipatory looks was directed to the tunnel exits. Substantially fewer looks fell onto the boxes. Unexpectedly, many looks were attracted by the tunnel window (the location where the chaser was last seen).

\hypertarget{inferential-statistics.}{%
\paragraph{Inferential statistics.}\label{inferential-statistics.}}

To further assess the pilot data and test our proposed analysis described in the main text, we ran two Bayesian mixed effect models as described in the main manuscript, the first using first look location as the dependent variable and the second using proportional looking score as the dependent variable. For the first look analysis, we defined the first look location as in the main text (corresponding roughly to the first look of 150 ms or more in the same AOI). We calculated the proportion looking (\emph{p}) to the correct AOI during the full 4000 ms anticipatory window by correct AOI looks / (correct AOI looks + incorrect AOI looks), excluding looks outside of either AOI. The anticipatory eye movement window was defined 120 ms after the first frame when the chaser had completely entered the tunnel and 120 ms after the chaser reappeared from the tunnel.\\
Because we wanted to ask if participants were attentive and could still make predictive looks at the end of the familiarization phase, we coded the trial number such that the last trial during the familiarization phase (the 8th in pilot 1) is set to 0, with trials 1 through 7 are coded as -7 to -1, respectively. We used the priors described in the main text of our analysis plan. Our base model was as follows, where measure refers to the dependent variable (either first look or the proportional looking score):\\
\(Measure ~ 1 + trial\_number + (trial\_number | lab) + (trial\_number | participant)\)
We fitted a reduced model for model comparison:
\(Measure ~ 0 + trial\_number + ( trial\_number | lab) + (trial\_number | participant)\)
We then calculated the Bayes factor, which we interpret as described in the main text.

\hypertarget{toddlers.}{%
\subparagraph{Toddlers.}\label{toddlers.}}

For the first look analysis, the intercept estimate was .44, (CrI95\% = 0.07, 0.80). This corresponds to a point estimate of a 61\% probability of the first look to be mapped onto the target as opposed to the distractor. The Bayes factor comparing the model with and without the intercept was 1.52, which is inconclusive by our criteria (Schönbrodt \& Wagenmakers, 2018). For the proportional looking score main model, the model estimate for the intercept was 0.16 (CrI95\% = 0.10, 0.23). This can be interpreted as a point estimate of a 66\% probability of looking at the target. The Bayes factor was 493.25, which was strong evidence in favor of the full model and which strongly suggests that toddlers looked more towards the target than towards the distractor during the anticipation period.

\hypertarget{adults.}{%
\subparagraph{Adults.}\label{adults.}}

For the first look analysis, we obtained a model estimate of 1.95 (CrI95\% = 1.42, 2.48). This corresponds to a probability of 88\% that the first look is to the target. The Bayes factor was \textgreater{} 1000, which was evidence in favor of the full model. For the Proportion Differential looking score analysis, the Bayes factor was also \textgreater{} 1000, which was evidence in favor of the full model. This suggested that adults had a higher proportion of looking at the target than chance level. The model estimate for the intercept was 0.46 (CrI95\% = 0.38, 0.54). Based on these analyses, it is clear that adults looked more to the target than the toddlers did, and it appears this was driven by Trials 5-8, as can be seen in Figure S2. Adults learn to anticipate the target and, on later trials, very rarely look at the distractor.

\hypertarget{discussion}{%
\subsubsection{Discussion}\label{discussion}}

Based on the first pilot, we drew the following conclusions: (1) Toddlers and adults show anticipation during the anticipatory period, and thus the paradigm seems successful at eliciting anticipation. (2) Over the course of eight trials, toddlers and adults remained attentive and showed anticipatory behavior even during the last trial of the familiarization phase. (3) Four familiarization trials seem to be sufficient and there do not appear to be strong additional benefits of running additional trials. Crucially, trials five to eight did not help to increase the overall anticipation rate for toddlers, as shown in Figure S2. Note that in the adults sample AL slightly increased after trial 4. We nonetheless decided to use four familiarization trials in the main study because we reasoned that it is more important to avoid fatigue or boredom in the toddlers sample than to get even higher anticipation rates for adults.
It is important to note that our decision to include 4 familiarization trials is based on (1) conceptual and practical methodological considerations also considering previous studies and (2) the pilot study results. Replication studies of Southgate, Senju, and Csibra (2007) pointed to issues with the familiarization phase and that the two trials of the original study might not be enough to familiarize toddlers with the scenario (e.g., Kampis \& Southgate, 2020; Schuwerk, Priewasser, Sodian, \& Perner, 2018). On the other hand, to avoid unnecessarily increasing the overall length of the task and to prevent poor anticipatory looking due to fatigue or boredom, we did not want to include too many familiarization trials. In the discussions preceding the pilot data analysis, we came to the conclusion that four trials reflect such an optimal trade-off. The pilot data results of the toddlers then supported this decision insofar as we observed a looking bias towards the correct location already in trials 1-4, without additional benefit of trials 5-8. Due to the exploratory nature of the pilot studies, we refrained from running inferential statistics in addition to the visual inspection of the first look and proportion looking data, as well as of the time series illustration, which all converged on this interpretation (see supplementary Figure S1 and S2).
The duration of the anticipatory period was set based on durations used in previous studies. Earlier studies found action outcome-contingent anticipatory looking with anticipatory phases ranging between approximately 2-3.5 seconds (Low \& Watts, 2013; Meristo et al., 2012; Surian \& Geraci, 2012; Thoermer et al., 2012). To make sure we are not losing anticipatory looks by cutting off too early, we decided to use a time period of 4 seconds. The pilot data showed no evidence for a decline in anticipatory looking towards the end of the anticipatory period (see time series plot in S2), which supported this decision.
Further, the distribution of looks in the anticipatory period helped us to evaluate the appropriateness of our AOI dimension, in particular whether restricting AOIs to the tunnel exits not including the adjacent box optimally captures goal-directed anticipatory looks. By increasing the AOI dimensions so that they cover both the tunnel exit and the box, we could potentially detect more goal-directed anticipatory looks. On the other hand, looks to the box cannot unambiguously be interpreted as anticipations of the chaser's upcoming action. Participants might look to the box simply because this is where the chasee is, anticipating that the chasee might jump out of the box again. Thus, we concluded that restricting our AOIs to the tunnel exits --the location where the chaser will reappear-- is the more conservative and more unambiguously interpretable measure of goal-directed action prediction. The result of our pilot study corroborated this strategy. The larger proportion of anticipatory looks was indeed directed to the tunnel exits and not to the boxes. Based on this finding, we concluded that using the tunnel exit AOIs is the sharper measure of goal-directed action predictions without a substantial loss of looks that could also reflect action predictions but are directed elsewhere (e.g., to the box).
An unexpected result of Pilot 1 was that during the anticipatory period, many fixations were attracted by the tunnel window where the agent was last seen. This was potentially problematic since looking at the window could lead to a reduced amount of anticipatory looks to the target/distractor AOIs. Initially, the window was added to the tunnel with the aim to increase AL (cf., Surian \& Franchin, 2020). But the results suggested that it may have been distracting, and so we removed the window for Pilot 2.

\hypertarget{pilot-2}{%
\subsection{Pilot 2}\label{pilot-2}}

To further hone our stimulus design, we conducted a second pilot. First, we removed the potentially distracting tunnel window from all trials in Pilot 2. Second, we tested another method to increase AL. We asked whether a chime as an arbitrary timing cue helps to elicit AL to the tunnel exits in (future) test trials in which the agent does not reappear at one of the tunnel exits (because these test trials stop after the end of the anticipatory phase without showing the agent's action outcome). To this end, we presented the first four familiarization trials showing the outcome associated with the chime, i.e., the chime announced the reappearance of the chaser, and four subsequent familiarization trials without an outcome, i.e., the chime sounded, but the chaser did not reappear. We reasoned that if participants learn in the first four trials that the chime indicates the chaser's reappearance, we should see an increase in AL right after the chime sounded. Further, this increase should also be observable in the last four trials in which the chaser does not reappear. Data collection for this pilot started in January 2020 and had to stop due to Covid-19 outbreak in March 2020.

\hypertarget{methods-1}{%
\subsubsection{Methods}\label{methods-1}}

\hypertarget{participants.-1}{%
\paragraph{Participants.}\label{participants.-1}}

A total of 12 healthy full-term toddlers participated in the second pilot study (6 males; Mean age = 24.15 months; range: 19.14 months to 26.05 months). One additional toddler was tested but excluded from the analyses because this toddler did not complete at least six trials. An additional one trial was excluded as the toddler did not look at least 32 seconds during this trial. We asked five labs to contribute a minimal sample size of four toddlers. Yet, data collection had to stop due to the Covid-19 outbreak.

\hypertarget{task-and-procedure.-1}{%
\paragraph{Task and Procedure.}\label{task-and-procedure.-1}}

The task and procedure were similar to Pilot 1. In this study, the following eye-tracking systems were used: Tobii T60 (one lab), Tobii T120 (one lab), EyeLink 1000 Plus (two labs), and Tobii Pro Spectrum (one lab). After the initial attention getter, participants were presented with the calibration check as in Pilot 1, eight familiarization trials and at the end, again the calibration check. The familiarization trials started by showing the same scene as in pilot 1, except that the window was removed from the tunnel. The trials differed in whether they displayed an outcome (i.e., the chaser exits the tunnel and the two agents rejoin) or not (i.e., trial stopped after the anticipatory period). The first four trials showed the outcome, the last four trials did not. Unlike in the first pilot, the chime now sounded the moment the chaser reappeared at one of the tunnel exits in the outcome trials. In the no outcome trials, the chime sounded the same moment, yet now the chaser did not appear. Again, the trials were presented in two pseudo-randomized orders {[}Order A: outcome (LR, LL, RR, RL), no outcome (LL, RL, LR, RR); Order B: outcome (RL, RR, LL, LR), no outcome (RR, LR, RL, LL{]}. Half of the participants in each lab group were randomly assigned to one of two orders.

\hypertarget{data-analysis.-1}{%
\paragraph{Data Analysis.}\label{data-analysis.-1}}

Data preprocessing was analogous to Pilot 1.

\hypertarget{results-and-discussion}{%
\subsubsection{Results and Discussion}\label{results-and-discussion}}

As can be seen in Figures S1 and S2, we found a similar pattern of results in both conditions of Pilot 2 (with outcome and without come) as we did in Pilot 1. We saw more looks directed towards the target than to the distractor. As described above, all trials in Pilot 2 lacked the tunnel window, whereas all trials in Pilot 1 included the tunnel window. Thus, we can assess the effect of the tunnel window by comparing Pilot 2 to Pilot 1. We found that the removal of the tunnel window did not appear to increase or decrease AL in Pilot 2 in any clear way. In fact, even after the removal of the window, a substantial amount of gaze was attracted towards the location where the window had been in Pilot 1 (for an illustration, see Figure S4). An explanation for this pattern of results is that not the window itself but its location in the center of the scene attracted visual attention. Previous research documented a central fixation bias in infants, toddlers and adults when viewing complex visual scenes (Tatler, 2007; van Renswoude et al., 2019).
By comparing the outcome and no outcome conditions in Pilot 2, we were able to assess whether the use of the chime helps AL. We did not find evidence that the chime helped to increase AL, and the majority of anticipatory looks to the tunnel exits happened before the chime sounded. As with Pilot 1, we ran a series of Bayesian mixed effect models to quantitatively evaluate anticipation. As we had a much smaller sample in Pilot 2, our Bayesian analyses were broadly inconclusive and did not favor either the full or null model. (Bayes factors fell between 0.1 and 3), suggesting that we did not have sufficient data to conclude whether the evidence is in favor of the full model or the simpler model. But, by comparing the results to the results of Pilot 1, we are confident that the results of Pilot 2 are qualitatively similar.

\hypertarget{conclusions}{%
\subsubsection{Conclusions}\label{conclusions}}

In both pilot studies we found that participants produced goal-directed action predictions. The combined analysis using AOIs around the tunnel exits revealed a looking bias towards the exit at which the chaser reappeared following their goal to catch the chasee. We are thus confident that participants clearly predicted the agent's action and did not just look at the chasee's location, anticipating something else. The changes of stimulus features in Pilot 2 did not affect AL rates. To reduce the complexity of the stimuli, we decided to use the stimuli without the tunnel window. Further, we removed the chime from the final version. In sum, we conclude that these novel stimuli sufficiently elicit goal-directed action predictions and are thus suited to serve as familiarization trials in the study described in the main text.

\hypertarget{s2.-further-supplemental-information-methods}{%
\section{S2. Further Supplemental Information: Methods}\label{s2.-further-supplemental-information-methods}}

\hypertarget{questionnaires-and-test-session-information}{%
\subsection{Questionnaires and test session information}\label{questionnaires-and-test-session-information}}

Using a questionnaire (filled out during the lab session or online for remote testing procedures) we will collect the following demographic information from the participating toddlers: gender, chronological age in days, nationality of the toddler, estimated proportion of language exposure, preterm/full-term status, current visual or hearing impairments, any known developmental concerns, information about siblings (number, gender, age), duration of time the toddler spends with caregivers and in day-care. From their caregivers the following information will be collected: gender, nationality, native language(s), level of education. For the adult sample, the following demographic information will be collected: gender, chronological age in years, and level of education.
Additionally, we collect the following information for each participant: name of lab the participant was tested in, academic status of the experimenter involved in the test session (e.g., volunteer, undergraduate, graduate, post-doctoral, professor), the type of eye-tracking apparatus used including sampling rate and screen dimensions (for eye-tracking procedures), date of testing, trial order condition the participant was assigned to, any procedural or technical error that occurred during the session and further reasons for exclusion, and the type of recruitment method the lab used. For the toddlers sample, we will additionally ask for the amount of experience the experimenter has in testing toddlers, and whether the toddler sat on the caregiver's lap or in a seat. The requested demographic information that is not used in the registered confirmatory and/or exploratory analyses of this study will be collected for further potential follow-up analyses in spin-off projects within the MB framework.

\hypertarget{stimuli}{%
\subsection{Stimuli}\label{stimuli}}

\hypertarget{general-scene-setup}{%
\subsubsection{General Scene Setup}\label{general-scene-setup}}

The depicted scene comprises an open space colored in blue. A horizontal picket fence divides the space into two sections (upper: approx. one third; lower: approx. two thirds). In the upper section, initially two animated, same-sized agents are seen: a brown bear (chaser) and a yellow mouse (chasee). The agents communicate using pseudo utterances. When they move, footsteps can be heard. The back of the upper section is formed by a wall with a small, central door through which the agents can enter and leave the scenario. Leaving through this door partially covers the agent, with the lower part of the body still visible. In the lower section of the scene, two identical brown boxes with moveable lids are located (one on the left and one on the right side). A white, centrally located, inverted Y-shaped tunnel connects both sides of the fence. One entrance is located in the upper section, while two identical exits are located in the lower section. Each exit in the lower section points towards the left or right box, respectively. The agents can move from the upper to the lower section of the scene by walking through the tunnel.

\hypertarget{familiarization-trials}{%
\subsubsection{Familiarization Trials}\label{familiarization-trials}}

All participants will view four familiarization trials. Each trial starts with the chaser and the chasee playing tag in the upper section of the scene. That is, the chasee runs off in a circle and is closely followed by the chaser (\textasciitilde4 s). When the chasee stops, the chaser catches up and they do a high five (\textasciitilde1 s). After separating again, the agents stand next to each other in front of the tunnel´s entrance (left or right position counterbalanced) (\textasciitilde3 s). Next, the chasee makes eye contact with the chaser (\textasciitilde2 s) and leaves for the tunnel. The chaser watches closely as the chasee walks towards the tunnel and enters it (\textasciitilde2 s). The chaser then positions itself centrally in front of the tunnel entrance (\textasciitilde4 s). While the chasee is walking through the tunnel for four seconds, there is a sound of footsteps. The footsteps cease when the chasee leaves the tunnel through one of the two exits (left or right, counterbalanced) in the lower section (\textasciitilde3 s). At this point, the chasee briefly stops, turns around and establishes eye contact with the chaser across the fence (\textasciitilde1 s). The chaser raises their hands to the mouth and shouts (\textasciitilde2 s). Next, the chasee continues towards the box at the tunnel exit (\textasciitilde1 s). The lid of the box opens (accompanied by a clap sound) and the chasee jumps into it - after which the lid of the box closes, again accompanied by a clap sound (\textasciitilde1 s). Then, the chaser walks towards the tunnel entrance (\textasciitilde2 s) and transits through the tunnel. While it is walking through the tunnel, footsteps sound (\textasciitilde4 s - anticipatory period). A chime is played in the moment in which the chaser exits the tunnel (cue for the approach phase of the chaser). After leaving the tunnel (\textasciitilde2 s), the chaser approaches the box in which the chasee is hiding and knocks on it (\textasciitilde2 s). Then, the chasee jumps out of the box (with a box opening clap sound) and the chaser and chasee do a high five (\textasciitilde4 s).

\hypertarget{test-trials}{%
\subsubsection{Test Trials}\label{test-trials}}

Test trials start with the same chasing sequence as in the familiarization trials. After doing a high five, chaser and chasee take their positions in front of the tunnel entrance. Next, the chasee makes eye contact with the chaser, leaves for the tunnel and enters it. From this point onwards, the events depend on the condition:
In the ignorance condition, after the chasee entered the tunnel (\textasciitilde12 s after start), the chaser exits through the door in the wall in the back (\textasciitilde4 s). The back of the chaser remains visible. While the chaser is away (for \textasciitilde8 s), the chasee walks through the tunnel (\textasciitilde4 s) and leaves through one of the exits (left or right, counterbalanced)(\textasciitilde2 s) and jumps into the respective box (\textasciitilde1 s). After approximately one second, while the chaser is still away, the chasee leaves this box A and tiptoes to the other box (\textasciitilde4 s). The chasee then jumps into box B and the lid closes (\textasciitilde1 s). In contrast to the familiarization trials, the chasee and the boxes make no sounds and no chime is played. After the hiding event has finished, the chaser returns through the door in the wall (\textasciitilde3 s) and enters the tunnel (\textasciitilde2 s). While the chaser is in the tunnel, footsteps are heard (\textasciitilde4 s). The video ends before the chaser exits the tunnel.
In the knowledge condition, the chaser remains on the scene in the upper section and positions itself centrally in front of the tunnel entrance (\textasciitilde2 s). Following the same sequence as in the ignorance condition, the chasee walks through the tunnel (\textasciitilde2 s), leaves it through one of the exits (left or right, counterbalanced) (\textasciitilde2 s) and hides in the respective box (\textasciitilde1 s). Next, in order to match the events of the ignorance condition, the chaser walks towards the door in the wall (\textasciitilde3 s) and disappears for approximately 1 seconds. Subsequently, they return to the initial position in front of the tunnel entrance (\textasciitilde3 s). In the meantime, the chasee did not move, so that the chaser did not miss any events while they were gone. Once the chaser returns it observes the chasee jump out of the first box (\textasciitilde1 s) and tiptoe to the second box (\textasciitilde4 s). Finally, the chasee jumps into the second box and the lid closes (\textasciitilde1 s). Like in the ignorance condition, the chasee and the boxes make no sound and no chime is played. The chaser enters the tunnel (\textasciitilde2 s) and footsteps sound (\textasciitilde4 s). Like in the ignorance condition, the video ends before the chaser exits the tunnel.

\hypertarget{trial-randomization}{%
\subsubsection{Trial randomization}\label{trial-randomization}}

The four combinations in familiarization were the following: started from the right and ended up in right box (RR); started from the right and ended up in left box (RL); started from the left and ended up in right box (LR); started from the left and ended up in left box (LL). The presentation of the familiarization trials will be counterbalanced in two pseudo-randomized orders (familiarization order A: Fam\_LR, Fam\_RR, Fam\_LL, Fam\_RL; familiarization order B: Fam\_RL, Fam\_LL, Fam\_LR, Fam\_RR). As with the familiarization trials, there will be four different parallel versions of the test trial for the knowledge and the ignorance condition, differing in the starting location of the chasee and the box the chasee ended up (Know\_RR, Know\_RL, Know\_LR, Know\_LL; Ig\_RR, Ig\_RL, Ig\_LR, Ig\_LL). Supplementary Table S2 lists the combinations that will be tested. Each lab signs up for one or two trial bins (16 trial combinations per bin) for each tested age group.

\hypertarget{general-lab-practices}{%
\subsection{General Lab Practices}\label{general-lab-practices}}

\hypertarget{training-of-research-assistants}{%
\subsubsection{Training of Research Assistants}\label{training-of-research-assistants}}

Each participating lab is responsible for maintaining the highest possible experimental standards, providing training practices for all experimenters and research assistants, and following detailed, written instructions to achieve uniformity and minimize variation across labs. Individual labs will document which experimenter(s) and research assistant(s) will test each participant. A questionnaire will serve to record and compare training practices. Greeting practices and instructions given to the participant/caregiver are marked down and standardized.

\hypertarget{reporting-of-technology-mishaps-and-participantcaregiver-behavior}{%
\subsubsection{Reporting of Technology Mishaps and Participant/Caregiver Behavior}\label{reporting-of-technology-mishaps-and-participantcaregiver-behavior}}

All labs are required to report anomalies, technical issues, concerns, and general comments on the protocol sheet. For toddler samples, concerns and general comments comprise the following: crying, fussiness, weariness, caregiver intervening (verbal or non-verbal, e.g., pointing), affecting or disrupting participation and/or looking behavior. Technical issues include problems that hinder, pause, or stop the stimulus presentation and/or eye-tracking recording.

\hypertarget{participant-exclusion}{%
\subsection{Participant exclusion}\label{participant-exclusion}}

Of the initial sample (toddlers: \emph{N} = 809, adults: \emph{N} = 805), participants will be excluded from the main confirmatory analyses if:
They did not complete the full experiment (toddlers: \emph{n} = 26, 3.21\%; adults: \emph{n} = 0, 0\%),
Participants' caregivers interfered with the procedure, e.g., by pointing at stimuli or talking to their toddler (toddlers: \emph{n} = 11, 1.36\%; adults: \emph{n} = 0, 0\%),
the experimenter made an error during testing that was relevant to the procedure (toddlers: \emph{n} = 11, 1.36\%; adults: \emph{n} = 6, 0.75\%),
technical problems occurred, e.g., data not saved, unable to calibrate eye-tracker, eye-tracker lost signal, data loss due to computer failure, computer crashed during recording (toddlers: \emph{n} = 69, 8.53\%; adults: \emph{n} = 61, 7.58\%).
The individual labs will determine whether and to which extent participant exclusion criteria 1-4 apply and add this information to the participant protocol sheet they provide. This set of exclusions will leave a total of 703 toddlers and 736 adults whose data will be analyzed. Of these, participants will be excluded sequentially if:
5. Their data were excluded due to missingness (see Preprocessing section) from more than one familiarization trial (toddlers: \emph{n} = 112, 13.84\%; adults: \emph{n} = 24, 2.98\%),
6. Their data from the first (critical) test trial were excluded due to missingness (toddlers: \emph{n} = 54, 6.67\%; adults: \emph{n} = 9, 1.12\%).
If multiple reasons for exclusion are applicable to a participant, the criteria will be assigned in the order above.

\newpage

\hypertarget{references}{%
\section{References}\label{references}}

\begingroup
\setlength{\parindent}{-0.5in}
\setlength{\leftskip}{0.5in}

\hypertarget{refs}{}
\begin{CSLReferences}{1}{0}
\leavevmode\vadjust pre{\hypertarget{ref-kampis2020altercentric}{}}%
Kampis, D., \& Southgate, V. (2020). Altercentric cognition: How others influence our cognitive processing. \emph{Trends in Cognitive Sciences}, \emph{24}(11), 945--959.

\leavevmode\vadjust pre{\hypertarget{ref-schuwerk2018robustness}{}}%
Schuwerk, T., Priewasser, B., Sodian, B., \& Perner, J. (2018). The robustness and generalizability of findings on spontaneous false belief sensitivity: A replication attempt. \emph{Royal Society Open Science}, \emph{5}(5), 172273.

\leavevmode\vadjust pre{\hypertarget{ref-southgate2007action}{}}%
Southgate, V., Senju, A., \& Csibra, G. (2007). Action anticipation through attribution of false belief by 2-year-olds. \emph{Psychological Science}, \emph{18}(7), 587--592.

\end{CSLReferences}

\endgroup


\end{document}
